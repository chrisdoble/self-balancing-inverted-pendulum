\documentclass{article}
\usepackage{amsmath} % For align*
\usepackage{circuitikz} % For circuit diagrams
\usepackage{bookmark} % For links
\usepackage{tikz} % For diagrams
\usepackage{siunitx} % For units

\hypersetup{
  colorlinks=true,
  linkcolor=blue,
  urlcolor=blue
}

\renewcommand{\vec}[1]{\boldsymbol{\mathbf{#1}}}

\begin{document}

\tableofcontents

\section{Introduction}

I recently built this self-balancing inverted pendulum on a cart. If you’ve ever tried to balance a broom on your hand you’ll know it’s tricky to keep it upright like this. Gravity is constantly pulling it down, so if it’s not perfectly vertical it’ll start to fall. The same thing is happening to the pendulum but — in the same way you would move your hand under a broom — the cart moves left and right to keep it upright.

In this video I’ll explain how I built this, starting with the math and physics, followed by the physical construction and code.

\section{Equations of Motion}

If we want to control this system we need to know how it behaves without any external input, so let’s derive its equations of motion. First, let's define our coordinate system and quantities in a diagram:

\begin{figure}[h]
  \centering
  \begin{tikzpicture}
    % x
    \draw (-2, 0.3) -- (-2, 0.7);
    \draw[dashed] (-2, 0.5) -- node[above] {$x$} (0, 0.5);

    % Cart
    \draw (0, 0) rectangle node {$m_1$} (2, 1);

    % Pendulum
    \draw (1, 1) -- (0.5, 3);
    \draw[dashed] (1, 1) -- (1, 3);
    \draw (0.37, 3.4) node {$m_2$} circle (0.4cm);

    % l
    \node at (0.5, 1.8) {$l$};

    % Theta
    \draw (1, 2.8) arc (90:117:1);
    \node at (0.8, 2.5) {$\theta$};

    % F
    \draw[->,thick] (2, 0.5) -- (3, 0.5) node[right] {$F$};

    % Axes
    \draw[->] (4, 0.5) -- (5, 0.5) node[right] {$x$};
    \draw[->] (4, 0.5) -- (4, 1.5) node[above] {$y$};
  \end{tikzpicture}
\end{figure}

You can see we have a rectangular cart of mass $m_1$ and it's a distance $x$ along the $x$-axis from some origin point. The cart is connected to a circular simple pendulum of length $l$ and mass $m_2$ which is at an angle $\theta$ from vertical.

With that out of the way, now we can determine the Lagrangian of the system. The Lagrangian is defined as the difference between the kinetic and potential energies of the system: $\mathcal{L} = T - U$. What are those energies?

Starting with the cart, it's constrained to the $x$-axis so its velocity has no $y$ component and its kinetic energy is \[T_\text{cart} = \frac{1}{2} m_1 \dot{x}^2.\] Its potential energy can't change so we might as well set it to $0$ \[U_\text{cart} = 0.\]

Next, the pendulum. It's $x$ and $y$ coordinates are \begin{align*}
  X & = x - l \sin \theta \\
  Y & = l \cos \theta.
\end{align*} We can differentiate these with respect to time to get the $x$ and $y$ components of its velocity \begin{align*}
  \dot{X} & = \dot{x} - l \dot{\theta} \cos \theta \\
  \dot{Y} & = -l \dot{\theta} \sin \theta.
\end{align*} This gives it a kinetic energy of \begin{align*}
  T_\text{pendulum} & = \frac{1}{2} m_2 V^2                                                                          \\
                    & = \frac{1}{2} m_2 (\dot{X}^2 + \dot{Y}^2)                                                      \\
                    & = \frac{1}{2} m_2 [(\dot{x} - l \dot{\theta} \cos \theta)^2 + (-l \dot{\theta} \sin \theta)^2] \\
                    & = \frac{1}{2} m_2 (\dot{x}^2 - 2 l \dot{\theta} \dot{x} \cos \theta + l^2 \dot{\theta}^2).
\end{align*} Unlike the cart, the pendulum's potential energy can change — as it rotates it moves up and down so its gravitational potential energy changes. If we say its potential energy is $0$ when it's $\ang{90}$ from vertical, then \[U_\text{pendulum} = m_2 g l \cos \theta.\]

Combining those energies gives a Lagrangian of \begin{align*}
  \mathcal{L} & = T - U                                                                                                                                     \\
              & = T_\text{cart} + T_\text{pendulum} - U_\text{cart} - U_\text{pendulum}                                                                     \\
              & = \frac{1}{2} m_1 \dot{x}^2 + \frac{1}{2} m_2 (\dot{x}^2 - 2 l \dot{\theta} \dot{x} \cos \theta + l^2 \dot{\theta}^2) - m_2 g l \cos \theta \\
              & = \frac{1}{2} (m_1 + m_2) \dot{x}^2 + \frac{1}{2} m_2 (l^2 \dot{\theta}^2 - 2 l \dot{\theta} \dot{x} \cos \theta) - m_2 g l \cos \theta.
\end{align*}

Now that we have the Lagrangian we can apply the Euler-Lagrange equation for the system's two coordinates $\theta$ and $x$. For $\theta$ we get \begin{align*}
  0 & = \frac{d}{d t} \frac{\partial \mathcal{L}}{\partial \dot{\theta}} - \frac{\partial \mathcal{L}}{\partial \theta}                 \\
    & = \frac{d}{d t} (m_2 l^2 \dot{\theta} - m_2 l \dot{x} \cos \theta) - m_2 l \dot{\theta} \dot{x} \sin \theta - m_2 g l \sin \theta \\
    & = l \ddot{\theta} - \ddot{x} \cos \theta - g \sin \theta
\end{align*} and for $x$ we get \begin{align*}
  0 & = \frac{d}{d t} \frac{\partial L}{\partial \dot{x}} - \frac{\partial \mathcal{L}}{\partial x} \\
    & = \frac{d}{d t} [(m_1 + m_2) \dot{x} - m_2 l \dot{\theta} \cos \theta]                        \\
    & = (m_1 + m_2) \ddot{x} - m_2 l \ddot{\theta} \cos \theta + m_2 l \dot{\theta}^2 \sin \theta.
\end{align*} Solving these two equations for $\ddot{\theta}$ and $\ddot{x}$ gives us our equations of motion \begin{align*}
  \ddot{\theta} & = \frac{(m_1 + m_2) g \sin \theta - m_2 l \dot{\theta}^2 \cos \theta \sin \theta}{l (m_1 + m_2) - m_2 l \cos^2 \theta} \\
  \ddot{x}      & = \frac{m_2 \sin 2 \theta - 2 m_2 l \dot{\theta}^2 \sin \theta}{2 m_1 + m_2 - m_2 \cos 2 \theta}.
\end{align*}

Now that we've got the equations of motion how do we know they're correct? One way is to solve them numerically and plot the solutions. This is a Mathematica notebook I created to do just that. You can see we define some constants like gravity, the length of the pendulum, etc. We define the initial conditions for the simulation — here the pendulum is starting $\ang{10}$ from vertical. We solve the equations of motion numerically, and we generate an animation of the solution which you can see if I evaluate the notebook.

The pendulum swings back and forth as we expect, but interestingly it also causes the cart to move back and forth. This model doesn't include air resistance or friction so it will continue forever, but we can tweak the initial conditions and see how that affects the animation. Let's make the initial angle $\ang{90}$. The cart still swings back and forth but the period seems to be shorter.

\end{document}