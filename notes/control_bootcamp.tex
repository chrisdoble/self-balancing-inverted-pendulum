\documentclass{article}
\usepackage{amsmath}
\usepackage{hyperref}

\hypersetup{
  colorlinks=true,
  linkcolor=blue,
  urlcolor=blue
}

\renewcommand{\vec}[1]{\boldsymbol{\mathbf{#1}}}
\newcommand{\dvec}[1]{\dot{\vec{#1}}}

\begin{document}

This document contains my notes on \href{https://www.youtube.com/playlist?list=PLMrJAkhIeNNR20Mz-VpzgfQs5zrYi085m}{Steve Brunton's Control Bootcamp video series}. Each section corresponds to the video of the same title.

\tableofcontents

\section{Overview}

\begin{itemize}
  \item \textbf{Passive controls} attempt to control a system passively, i.e. they are built into the system and don't vary based on observations of the system.

  \item \textbf{Active controls} attempt to control a system actively, i.e. they change their behaviour based on observations of the system.

  \item \textbf{Open-loop controllers} don't observe the output of the system — their inputs are predetermined. One downside of this approach is that you may unnecessarily input energy into the system when it already has the desired output.

  \item \textbf{Closed-loop controllers} observe the output of the system to determine their inputs. They have several benefits over open-loop controllers:

        \begin{itemize}
          \item They handle uncertainty in the system, e.g. if you don't always know how the system will respond, or if your model isn't completely accurate.

          \item They handle disturbances in the system, e.g. if someone pushes a self-balancing inverted pendulum. It may not be possible to account for these in the model.

          \item They can be more energy efficient than open-loop controllers, i.e. they don't have the downside mentioned above.
        \end{itemize}

  \item The mathematical model used in this course is a state-space based system of linear differential equations \[\dvec{x} = \vec{A} \vec{x}\] where $\vec{x}$ is the \textbf{state vector} — a vector containing all the system's values of interest — and $\dvec{x}$ are their rates of change at a time $t$.

  \item The solution to the above equation is \[\vec{x}(t) = e^{\vec{A} t} \vec{x}(0)\] and the eigenvalues of $\vec{A}$ can be used to determine the stability of the system, e.g. if they all have negative real components the system is stable.

  \item Control is introduced to the system by modifying the equation to \[\dvec{x} = \vec{A} \vec{x} + \vec{B} \vec{u}\] where $\vec{B}$ is a coefficient matrix and $\vec{u}$ is the input to the system.

  \item If we make the input to the system \[\vec{u} = -\vec{K} \vec{x}\] then \begin{align*}
          \dvec{x} & = \vec{A} \vec{x} - \vec{B} \vec{K} \vec{x} \\
                   & = (\vec{A} - \vec{B} \vec{K}) \vec{x}
        \end{align*} and now it is the eigenvalues of $\vec{A} - \vec{B} \vec{K}$ that determine the stability of the system, i.e. we can make an unstable system stable by appropriate choice of inputs.
\end{itemize}

\section{Linear Systems}

\begin{itemize}
  \item The solution to the linear system of differential equations \[\dvec{x} = \vec{A} \vec{x}\] is \[\vec{x}(t) = e^{\vec{A} t} \vec{x}_0\] where \begin{align*}
          e^{\vec{A} t} & = \sum_{k = 0}^\infty \frac{(\vec{A} t)^k}{k!}                                       \\
                        & = \vec{I} + \vec{A} t + \frac{\vec{A}^2 t^2}{2!} + \frac{\vec{A}^3 t^3}{3!} + \ldots
        \end{align*}

  \item If a matrix $\vec{A}$ has eigenvalues $\lambda_1, \ldots, \lambda_n$ and eigenvectors $\vec{k}_1, \ldots, \vec{k}_n$ then \[\vec{A} \vec{T} = \vec{T} \vec{D}\] or \[\vec{T}^{-1} \vec{A} \vec{T} = \vec{D}\] where \[\vec{T} = \begin{pmatrix}
            \vec{k}_1 & \cdots & \vec{k}_n
          \end{pmatrix}\] and \[\vec{D} = \begin{pmatrix}
            \lambda_1 &        & 0         \\
                      & \ddots &           \\
            0         &        & \lambda_n
          \end{pmatrix}.\]

  \item Using the above gives \[e^{\vec{A} t} = \vec{T} e^{\vec{D} t} \vec{T}^{-1}\] which is simpler because $e^{\vec{D} t}$ is easy to calculate.

  \item If we define $\vec{z}$ to be $\vec{x}$ transformed to the eigenvector basis, i.e. \[\vec{x} = \vec{T} \vec{z}\] and \[\dvec{x} = \vec{T} \dvec{z}\] then \begin{align*}
          \vec{z}  & = \vec{T}^{-1} \vec{x}                 \\
          \dvec{z} & = \vec{T}^{-1} \dvec{x}                \\
                   & = \vec{T}^{-1} \vec{A} \vec{x}         \\
                   & = \vec{T}^{-1} \vec{A} \vec{T} \vec{z} \\
                   & = \vec{D} \vec{z}
        \end{align*} where $\vec{D}$ is the diagonal matrix consisting of the eigenvalues of $\vec{A}$. The solution to this equation is \[\vec{z}(t) = e^{\vec{D} t} \vec{z}_0\] where \[e^{\vec{D} t} = \begin{pmatrix}
            e^{\lambda_1 t} &        & 0               \\
                            & \ddots &                 \\
            0               &        & e^{\lambda_n t}
          \end{pmatrix},\] i.e. the equations are uncoupled such that \begin{align*}
          z_1(t) & = z_{0,1} e^{\lambda_1 t}  \\
          \vdots                              \\
          z_n(t) & = z_{0,n} e^{\lambda_n t}.
        \end{align*}

  \item The equivalence between the original and eigenvector coordinates can be seen by manipulating the solution \begin{align*}
          \vec{x}(t) & = e^{\vec{A} t} \vec{x}_0                      \\
                     & = \vec{T} e^{\vec{D} t} \vec{T}^{-1} \vec{x}_0 \\
                     & = \vec{T} e^{\vec{D} t} \vec{z}_0              \\
                     & = \vec{T} \vec{z}(t).
        \end{align*}
\end{itemize}

\end{document}