\documentclass{article}
\usepackage{amsmath} % For align*
\usepackage{circuitikz} % For circuit diagrams
\usepackage{tikz} % For diagrams

\renewcommand{\vec}[1]{\boldsymbol{\mathbf{#1}}}

\begin{document}

\tableofcontents

\section{Equations of Motion}

\begin{figure}[h]
  \centering
  \begin{tikzpicture}
    % x
    \draw (-2, 0.3) -- (-2, 0.7);
    \draw[dashed] (-2, 0.5) -- node[above] {$x$} (0, 0.5);

    % Cart
    \draw (0, 0) rectangle node {M} (2, 1);

    % Pendulum
    \draw (1, 1) -- (0.5, 3);
    \draw[dashed] (1, 1) -- (1, 3);
    \draw (0.39, 3.3) node {m} circle (0.3cm);

    % l
    \node at (0.5, 1.8) {$l$};

    % Theta
    \draw (1, 2.8) arc (90:117:1);
    \node at (0.8, 2.5) {$\theta$};

    % F
    \draw[->,thick] (2, 0.5) -- (3, 0.5) node[right] {$F$};

    % Axes
    \draw[->] (4, 0) -- (5, 0) node[right] {$x$};
    \draw[->] (4, 0) -- (4, 1) node[above] {$y$};
  \end{tikzpicture}
\end{figure}

\begin{itemize}
  \item A cart of mass $M$ is constrained to move along the $x$ axis with its distance from an arbitrary point on the $x$ axis denoted $x$. A driving force of magnitude $F$ is applied to the cart in the $x$ direction. A simple pendulum consisting of a mass $m$ and a massless rod of length $l$ is connected to the cart with its angle from the positive $y$ axis denoted $\theta$.

  \item The kinetic energy of the cart is \[T_\text{cart} = \frac{1}{2} M \dot{x}^2.\]

  \item The $x$ and $y$ coordinates of the pendulum are \begin{align*}
          X & = x - l \sin \theta \\
          Y & = l \cos \theta,
        \end{align*} thus its $x$ and $y$ velocities are \begin{align*}
          \dot{X} & = \dot{x} - l \dot{\theta} \cos \theta \\
          \dot{Y} & = -l \dot{\theta} \sin \theta
        \end{align*} and its kinetic energy is \begin{align*}
          T_\text{pendulum} & = \frac{1}{2} m v^2                                                                          \\
                            & = \frac{1}{2} (\dot{X}^2 + \dot{Y}^2)                                                        \\
                            & = \frac{1}{2} m [(\dot{x} - l \dot{\theta} \cos \theta)^2 + (-l \dot{\theta} \sin \theta)^2] \\
                            & = \frac{1}{2} m (\dot{x}^2 - 2 l \dot{x} \dot{\theta} \cos \theta + l^2 \dot{\theta}^2).
        \end{align*}

  \item The total kinetic energy of the system is \begin{align*}
          T & = T_\text{cart} + T_\text{pendulum}                                                                          \\
            & = \frac{1}{2} (m + M) \dot{x}^2 + \frac{1}{2} m (l^2 \dot{\theta}^2 - 2 l \dot{x} \dot{\theta} \cos \theta).
        \end{align*}

  \item The potential energy of the system is equal to the gravitational potential energy of the pendulum. If its potential energy is $0$ when $\theta = \frac{\pi}{2}$ then \[U = m g l \cos \theta.\]

  \item The Lagrangian of the system is \begin{align*}
          \mathcal{L} & = T - U                                                                                                                          \\
                      & = \frac{1}{2} (m + M) \dot{x}^2 + \frac{1}{2} m (l^2 \dot{\theta}^2 - 2 l \dot{x} \dot{\theta} \cos \theta) - m g l \cos \theta.
        \end{align*}

  \item By d'Alembert's principle the generalized forces associated with the $\theta$ and $x$ coordinates are $0$ and $F$, respectively.

  \item The Euler-Lagrange equation for the $\theta$ coordinate is \begin{align*}
          \frac{d}{d t} \frac{\partial \mathcal{L}}{\partial \dot{\theta}} - \frac{\partial \mathcal{L}}{\partial \theta}         & = 0  \\
          \frac{d}{d t} (m l^2 \dot{\theta} - m l \dot{x} \cos \theta) - m l \dot{x} \dot{\theta} \sin \theta - m g l \sin \theta & = 0  \\
          l \ddot{\theta} - \ddot{x} \cos \theta - g \sin \theta                                                                  & = 0.
        \end{align*}

  \item The Euler-Lagrange equation for the $x$ coordinate is \begin{align*}
          \frac{d}{d t} \frac{\partial \mathcal{L}}{\partial \dot{x}} - \frac{\partial \mathcal{L}}{\partial x} & = F  \\
          \frac{d}{d t} [(m + M) \dot{x} - m l \dot{\theta} \cos \theta]                                        & = F  \\
          (m + M) \ddot{x} - m l \ddot{\theta} \cos \theta + m l \dot{\theta}^2 \sin \theta                     & = F.
        \end{align*}

  \item Solving these equations for $\ddot{\theta}$ and $\ddot{x}$ gives \[\ddot{\theta} = \frac{(m + M) g \sin \theta + F \cos \theta - m l \dot{\theta}^2 \cos \theta \sin \theta}{l (m + M) - m l \cos^2 \theta}\] and \[\ddot{x} = \frac{2 F + m g \sin 2 \theta - 2 m l \dot{\theta}^2 \sin \theta}{m + 2 M - m \cos 2 \theta}.\]
\end{itemize}

\section{Linearization, Stability, and Controllability} \label{controllability}

\begin{itemize}
  \item The state vector for this system is \[\begin{pmatrix}
            \theta       \\
            \dot{\theta} \\
            x            \\
            \dot{x}
          \end{pmatrix}.\]

  \item The fixed point about which the system will be linearized is \[\begin{pmatrix}
            0 \\
            0 \\
            0 \\
            0
          \end{pmatrix}.\]

  \item The $\vec{A}$ matrix is equal to the Jacobian matrix evaluated at the fixed point \[\vec{A} = \begin{pmatrix}
            0                     & 1 & 0 & 0 \\
            \frac{g (m + M)}{l M} & 0 & 0 & 0 \\
            0                     & 0 & 0 & 1 \\
            \frac{g m}{M}         & 0 & 0 & 0
          \end{pmatrix}.\]

  \item The non-zero eigenvalues of $\vec{A}$ are \[\pm \sqrt{\frac{g (m + M)}{l M}}.\] Because one of these has a positive real part the system is unstable.

  \item Rearranging the equations of motion to find the coefficients of $F$ gives \[\ddot{\theta} = f(\theta, \dot{\theta}) + \frac{\cos \theta}{l (m + M) - m l \cos^2 \theta} F\] and \[\ddot{x} = g(\theta, \dot{\theta}) + \frac{2}{m + 2 M - m \cos 2 \theta} F.\] Using the small angle approximation for $\cos$ gives \[\ddot{\theta} = f(\theta, \dot{\theta}) + \frac{1}{l M} F\] and \[\ddot{x} = g(\theta, \dot{\theta}) + \frac{1}{M} F\] resulting in the $\vec{B}$ matrix \[\begin{pmatrix}
            0             \\
            \frac{1}{l M} \\
            0             \\
            \frac{1}{M}
          \end{pmatrix}.\]

  \item The controllability matrix \[C = \begin{pmatrix}
            \vec{B} & \vec{A} \vec{B} & \vec{A}^2 \vec{B} & \vec{A}^3 \vec{B}
          \end{pmatrix}\] has full rank ($4$) so the system is controllable via the force $F$ on the cart.

  \item The ideal state feedback gains matrix $\vec{K}$ can be determined using Mathematica's \texttt{LQRegulatorGains} function.

  \item Thus, the force to apply to the cart when the system is in state $\vec{x}$ is \[F = u = -\vec{K} \vec{x}.\]
\end{itemize}

\section{Motor Control}

\begin{itemize}
  \item Using the equations above we can calculate the force to apply to the cart when the system is in a given state. However, the motor (which applies a force to the cart) is controlled via pulse-width modulation (PWM) and it's not clear what force results from a particular duty cycle.

  \item The motor can be electrically modelled by the circuit

        \begin{figure}[h]
          \centering
          \begin{circuitikz}[american voltages]
            \draw (0, 0) to [V, invert, l=$V_\text{in}$] (0, 2)
            to [resistor, l=$R$] (3, 2)
            to [inductor, l=$L$] (5, 2)
            to [V, l=$V_\text{motor}$] (5, 0)
            to (0, 0);
          \end{circuitikz}
        \end{figure}

        where $V_\text{in}$ is the (effective) voltage delivered to the motor via PWM, $R$ is the resistance of the motor's windings, $L$ is the inductance of the motor's windings, and $V_\text{motor}$ is the back EMF generated as the motor is spinning.

  \item If we assume that the back EMF generated by the motor is proportional to its angular velocity, i.e. $V_\text{motor} = K_\text{b} \omega$, then Kirchoff's voltage law gives \begin{align*}
          V_\text{in} - I R - \frac{d I}{d t} L - V_\text{motor}    & = 0  \\
          V_\text{in} - I R - \frac{d I}{d t} L - K_\text{b} \omega & = 0.
        \end{align*}

  \item The motor can be mechanically modelled as a rotating cylinder with equation of motion \begin{align*}
          \tau_\text{m} + \tau_\text{r} & = J \alpha \\
          K_\text{t} I - b \omega       & = J \alpha
        \end{align*} where $\tau_m$ is the torque due to the motor which is proportional to the current, $\tau_\text{r}$ is the resistive torque due to air resistance, friction, etc. which is proportional to the angular velocity $\omega$, $J$ is the moment of inertia, and $\alpha$ is the angular acceleration.

  \item Rearranging the last equation for $I$ we find \[I = \frac{1}{K_\text{t}} (b \omega + J \alpha)\] and \[\frac{d I}{d t} = \frac{1}{K_\text{t}} (b \alpha + J \dot{\alpha}).\]

  \item Substituting this into the electrical equation above gives \begin{align*}
          V_\text{in} - \frac{R}{K_\text{t}} (b \omega + J \alpha) - \frac{L}{K_\text{t}} (b \alpha + J \dot{\alpha}) - K_\text{b} \omega                                                       & = 0 \\
          V_\text{in} - \left( \frac{b R}{K_\text{t}} + K_\text{b} \right) \omega - \left( \frac{J R}{K_\text{t}} + \frac{b L}{K_\text{t}} \right) \alpha - \frac{J L}{K_\text{t}} \dot{\alpha} & = 0
        \end{align*} or \[A \ddot{\omega} + B \dot{\omega} + C \omega + V_\text{in} = 0\] where $A$, $B$, and $C$ are constants.

  \item The general solution to this equation is \[\omega = A V_\text{in} + B e^{C t} + D e^{E t}\] where $A$, $B$, $C$, $D$, and $E$ are different constants from above.

  \item The equation can be simplified by assuming $D = 0$ \[\omega = A V_\text{in} + B e^{C t}.\]

  \item In reality $\omega$ quickly changes from $0$ to some equilibrium value that has the same sign as $V_\text{in}$. That being the case, we know:

        \begin{itemize}
          \item $C$ must be negative otherwise $\omega$ would increase without bound,

          \item $A$ must be positive as the equilibrium value has the same sign as $V_\text{in}$, and

          \item $B = -A V_\text{in}$ because $\omega(0) = 0$.
        \end{itemize}

        Finally, the equation can be multiplied by the radius of the timing pulley $R$ to give the velocity of the cart $v$. This gives \[v = A V_\text{in} (1 - e^{-B t})\] where $A, B > 0$ are constants that can be determined from experimental data.

  \item Differentiating this equation with respect to time gives \[a = A B V_\text{in} e^{-B t}\] which can be equated with the time-independent equation for $a = R \dot{\omega}$ \begin{align*}
          A B V_\text{in} e^{-B t} & = C V_\text{in} + D v                               \\
                                   & = C V_\text{in} + D [A V_\text{in} (1 - e^{-B t})]  \\
                                   & = (A D + C) V_\text{in} - A D V_\text{in} e^{-B t}.
        \end{align*} From this we can see $A B = -A D$ or $D = - B$ and $A D + C = 0$ or $C = A B$. Once $A$ and $B$ are determined from experimental data we can determine $C$ and $D$ and thus the time-independent equation for $a$.

  \item The time-independent equation for $a$ can be multiplied by the mass of the cart $M$ to find the force on the cart $F$. This equation can then be rearranged to find the voltage $V_\text{in}$ required to exert a force $F$ on the cart when it has velocity $v$ \begin{align*}
          a           & = C V_\text{in} + D v                           \\
          F           & = M (C V_\text{in} + D v)                       \\
          V_\text{in} & = \frac{1}{C} \left( \frac{F}{M} - D v \right).
        \end{align*} This equation bridges the control equations of section \ref{controllability} and the motor equations of this section.
\end{itemize}

\end{document}