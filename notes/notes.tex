\documentclass{article}
\usepackage{amsmath} % For align*
\usepackage{tikz} % For diagrams

\renewcommand{\vec}[1]{\boldsymbol{\mathbf{#1}}}

\begin{document}

\tableofcontents

\section{Equations of Motion}

\begin{figure}[h]
  \centering
  \begin{tikzpicture}
    % x
    \draw (-2, 0.3) -- (-2, 0.7);
    \draw[dashed] (-2, 0.5) -- node[above] {$x$} (0, 0.5);

    % Cart
    \draw (0, 0) rectangle node {M} (2, 1);

    % Pendulum
    \draw (1, 1) -- (0.5, 3);
    \draw[dashed] (1, 1) -- (1, 3);
    \draw (0.39, 3.3) node {m} circle (0.3cm);

    % l
    \node at (0.5, 1.8) {$l$};

    % Theta
    \draw (1, 2.8) arc (90:117:1);
    \node at (0.8, 2.5) {$\theta$};

    % F
    \draw[->,thick] (2, 0.5) -- (3, 0.5) node[right] {$F$};

    % Axes
    \draw[->] (4, 0) -- (5, 0) node[right] {$x$};
    \draw[->] (4, 0) -- (4, 1) node[above] {$y$};
  \end{tikzpicture}
\end{figure}

\begin{itemize}
  \item A cart of mass $M$ is constrained to move along the $x$ axis with its distance from an arbitrary point on the $x$ axis denoted $x$. A driving force of magnitude $F$ is applied to the cart in the $x$ direction. A simple pendulum consisting of a mass $m$ and a massless rod of length $l$ is connected to the cart with its angle from the positive $y$ axis denoted $\theta$.

  \item The kinetic energy of the cart is \[T_\text{cart} = \frac{1}{2} M \dot{x}^2.\]

  \item The $x$ and $y$ coordinates of the pendulum are \begin{align*}
          X & = x - l \sin \theta \\
          Y & = l \cos \theta,
        \end{align*} thus its $x$ and $y$ velocities are \begin{align*}
          \dot{X} & = \dot{x} - l \dot{\theta} \cos \theta \\
          \dot{Y} & = -l \dot{\theta} \sin \theta
        \end{align*} and its kinetic energy is \begin{align*}
          T_\text{pendulum} & = \frac{1}{2} m v^2                                                                          \\
                            & = \frac{1}{2} (\dot{X}^2 + \dot{Y}^2)                                                        \\
                            & = \frac{1}{2} m [(\dot{x} - l \dot{\theta} \cos \theta)^2 + (-l \dot{\theta} \sin \theta)^2] \\
                            & = \frac{1}{2} m (\dot{x}^2 - 2 l \dot{x} \dot{\theta} \cos \theta + l^2 \dot{\theta}^2).
        \end{align*}

  \item The total kinetic energy of the system is \begin{align*}
          T & = T_\text{cart} + T_\text{pendulum}                                                                          \\
            & = \frac{1}{2} (m + M) \dot{x}^2 + \frac{1}{2} m (l^2 \dot{\theta}^2 - 2 l \dot{x} \dot{\theta} \cos \theta).
        \end{align*}

  \item The potential energy of the system is equal to the gravitational potential energy of the pendulum. If its potential energy is $0$ when $\theta = \frac{\pi}{2}$ then \[U = m g l \cos \theta.\]

  \item The Lagrangian of the system is \begin{align*}
          \mathcal{L} & = T - U                                                                                                                          \\
                      & = \frac{1}{2} (m + M) \dot{x}^2 + \frac{1}{2} m (l^2 \dot{\theta}^2 - 2 l \dot{x} \dot{\theta} \cos \theta) - m g l \cos \theta.
        \end{align*}

  \item By d'Alembert's principle the generalized forces associated with the $\theta$ and $x$ coordinates are $0$ and $F$, respectively.

  \item The Euler-Lagrange equation for the $\theta$ coordinate is \begin{align*}
          \frac{d}{d t} \frac{\partial \mathcal{L}}{\partial \dot{\theta}} - \frac{\partial \mathcal{L}}{\partial \theta}         & = 0  \\
          \frac{d}{d t} (m l^2 \dot{\theta} - m l \dot{x} \cos \theta) - m l \dot{x} \dot{\theta} \sin \theta - m g l \sin \theta & = 0  \\
          l \ddot{\theta} - \ddot{x} \cos \theta - g \sin \theta                                                                  & = 0.
        \end{align*}

  \item The Euler-Lagrange equation for the $x$ coordinate is \begin{align*}
          \frac{d}{d t} \frac{\partial \mathcal{L}}{\partial \dot{x}} - \frac{\partial \mathcal{L}}{\partial x} & = F  \\
          \frac{d}{d t} [(m + M) \dot{x} - m l \dot{\theta} \cos \theta]                                        & = F  \\
          (m + M) \ddot{x} - m l \ddot{\theta} \cos \theta + m l \dot{\theta}^2 \sin \theta                     & = F.
        \end{align*}

  \item Solving these equations for $\ddot{\theta}$ and $\ddot{x}$ gives \[\ddot{\theta} = \frac{(m + M) g \sin \theta + F \cos \theta - m l \dot{\theta}^2 \cos \theta \sin \theta}{l (m + M) - m l \cos^2 \theta}\] and \[\ddot{x} = \frac{2 F + m g \sin 2 \theta - 2 m l \dot{\theta}^2 \sin \theta}{m + 2 M - m \cos 2 \theta}.\]
\end{itemize}

\section{Linearization, Stability, and Controllability}

\begin{itemize}
  \item The state vector for this system is \[\begin{pmatrix}
            \theta       \\
            \dot{\theta} \\
            x            \\
            \dot{x}
          \end{pmatrix}.\]

  \item The fixed point about which the system will be linearized is \[\begin{pmatrix}
            0 \\
            0 \\
            0 \\
            0
          \end{pmatrix}.\]

  \item The $\vec{A}$ matrix is equal to the Jacobian matrix evaluated at the fixed point \[\vec{A} = \begin{pmatrix}
            0                     & 1 & 0 & 0 \\
            \frac{g (m + M)}{l M} & 0 & 0 & 0 \\
            0                     & 0 & 0 & 1 \\
            \frac{g m}{M}         & 0 & 0 & 0
          \end{pmatrix}.\]

  \item The non-zero eigenvalues of $\vec{A}$ are \[\pm \sqrt{\frac{g (m + M)}{l M}}.\] Because one of these has a positive real part the system is unstable.

  \item Rearranging the equations of motion to find the coefficients of $F$ gives \[\ddot{\theta} = f(\theta, \dot{\theta}) + \frac{\cos \theta}{l (m + M) - m \cos^2 \theta} F\] and \[\ddot{x} = g(\theta, \dot{\theta}) + \frac{2}{m + 2 M - m \cos 2 \theta} F.\] Using the small angle approximation for $\cos$ gives \[\ddot{\theta} = f(\theta, \dot{\theta}) + \frac{1}{l M} F\] and \[\ddot{x} = g(\theta, \dot{\theta}) + \frac{1}{M} F\] resulting in the $\vec{B}$ matrix \[\begin{pmatrix}
            0             \\
            \frac{1}{l M} \\
            0             \\
            \frac{1}{M}
          \end{pmatrix}.\]

  \item The controllability matrix \[C = \begin{pmatrix}
            \vec{B} & \vec{A} \vec{B} & \vec{A}^2 \vec{B} & \vec{A}^3 \vec{B}
          \end{pmatrix}\] has full rank ($4$) so the system is controllable via the force $F$ on the cart.

  \item The ideal state feedback gains matrix $\vec{K}$ can be determined using Mathematica's \texttt{LQRegulatorGains} function.
\end{itemize}

\end{document}